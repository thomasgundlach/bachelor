% !TEX TS-program = pdflatex
% !TEX encoding = UTF-8 Unicode

% This is a simple template for a LaTeX document using the "article" class.
% See "book", "report", "letter" for other types of document.

\documentclass[10pt]{scrartcl} % use larger type; default would be 10pt
%\documentclass[10pt,landscape]{scrartcl} % use larger type; default would be 10pt

\usepackage[utf8]{inputenc} % set input encoding (not needed with XeLaTeX)

%%% Examples of Article customizations
% These packages are optional, depending whether you want the features they provide.
% See the LaTeX Companion or other references for full information.

%%% PAGE DIMENSIONS
\usepackage{geometry} % to change the page dimensions
\geometry{a3paper} % or letterpaper (US) or a5paper or....
\geometry{margin=0.0cm} % for example: '2in' change the margins to 2 inches all round
% \geometry{landscape} % set up the page for landscape
%   read geometry.pdf for detailed page layout information

\usepackage{amssymb}

\usepackage{graphicx} % support the \includegraphics command and options

% \usepackage[parfill]{parskip} % Activate to begin paragraphs with an empty line rather than an indent

%%% PACKAGES
\usepackage{booktabs} % for much better looking tables
\usepackage{array} % for better arrays (eg matrices) in maths
\usepackage{paralist} % very flexible & customisable lists (eg. enumerate/itemize, etc.)
\usepackage{verbatim} % adds environment for commenting out blocks of text & for better verbatim
\usepackage{subfig} % make it possible to include more than one captioned figure/table in a single float
% These packages are all incorporated in the memoir class to one degree or another...

%%% HEADERS & FOOTERS
\usepackage{fancyhdr} % This should be set AFTER setting up the page geometry
\pagestyle{fancy} % options: empty , plain , fancy
\renewcommand{\headrulewidth}{0pt} % customise the layout...
\lhead{}\chead{}\rhead{}
\lfoot{}\cfoot{\thepage}\rfoot{}

%%% SECTION TITLE APPEARANCE
\usepackage{sectsty}
\allsectionsfont{\sffamily\mdseries\upshape} % (See the fntguide.pdf for font help)
% (This matches ConTeXt defaults)

%%% ToC (table of contents) APPEARANCE
\usepackage[nottoc,notlof,notlot]{tocbibind} % Put the bibliography in the ToC
\usepackage[titles,subfigure]{tocloft} % Alter the style of the Table of Contents
\renewcommand{\cftsecfont}{\rmfamily\mdseries\upshape}
\renewcommand{\cftsecpagefont}{\rmfamily\mdseries\upshape} % No bold!
\usepackage[colorlinks]{hyperref}

\usepackage{tabularx}         % new column specifier
\usepackage{longtable}        % long tables
\usepackage{ltxtable}         % provides the functionality of longtable

\usepackage{pdflscape}

% timeplan
\PassOptionsToPackage{rgb,svgnames}{xcolor}
\usepackage{pgfgantt}

% Tabellenbefehle:
%
\newenvironment{titletab}{\sffamily\selectfont\tabularx}{\endtabularx}
% Neuer Kollumnentyp für array mit Flattersatz
\newcolumntype{Y}{>{\small\raggedright\arraybackslash}X}
% p{} mit Flattersatz!
\newcolumntype{v}[1]{>{\small\raggedright\arraybackslash}p{#1}}
% l und r mit schmaler Schrift
\newcolumntype{L}{>{\small\arraybackslash}l}
\newcolumntype{R}{>{\small\arraybackslash}r}
%

%%% END Article customizations

%%% The "real" document content comes below...

\title{}
\subtitle{Timetable Thesis Jan Thomas Gundlach}
%\author{jgu}
%\date{} % Activate to display a given date or no date (if empty),
         % otherwise the current date is printed

\begin{document}
%\maketitle

\section*{Timetable Thesis Jan Thomas Gundlach}
\subsection*{Visualisierung von Aufgabenrelation}
\subsubsection*{Anhand archivierter Daten der Prozessengine Automation Automic}
\resizebox{!}{.83\height}{
%\begin{landscape}
\begin{ganttchart}[%Specs
     y unit title=0.5cm,
     y unit chart=0.7cm,
     vgrid,hgrid,
     title height=1,
%     title/.style={fill=none},
     title label font=\bfseries\footnotesize,
     bar/.style={fill=yellow},
     bar height=0.7,
%   progress label text={},
     group right shift=0,
     group top shift=0.7,
     group height=.3,
     group peaks width={0.2},
     inline]{1}{10}
\gantttitle{Juni 21}{2}
\gantttitle{Juli 21}{4}
\gantttitle{August 21}{4}\\
\gantttitlelist{25,...,26}{1}
\gantttitlelist{27,...,30}{1}
\gantttitlelist{31,...,34}{1}\\
\ganttgroup{Bachelor-Arbeit}{1}{10} \\
\ganttgroup{Vorbereitung}{1}{4} \\
%\ganttmilestone[inline=false]{Aufstellung Zeitplan}{0} \\
%\ganttbar[progress=0,inline=false]{Kennenlernen und Idee besprechen}{1}{1} \\
%\ganttmilestone[inline=false]{Bachelor-Arbeit beantragen}{2} \\
\ganttbar[progress=70,inline=false]{Einleitung}{3}{4} \\
\ganttbar[progress=20,inline=false]{Grundlagen}{4}{5} \\
\ganttmilestone[inline=false]{Review}{5} \\
\ganttgroup{Entwicklung}{6}{8} \\
\ganttbar[progress=0,inline=false]{Analyse}{6}{6} \\
%\ganttmilestone[inline=false]{Review}{5} \\
\ganttbar[progress=0,inline=false]{Konzept}{7}{7} \\
\ganttbar[progress=0,inline=false]{Implementierung}{7}{8} \\
\ganttmilestone[inline=false]{Review}{7} \\
\ganttgroup{Finalisierung}{9}{10} \\
\ganttbar[progress=0,inline=false]{Zusammenfassung}{9}{9} \\
\ganttmilestone[inline=false]{Review}{9} \\
\ganttbar[progress=0,inline=false]{Schreiben}{1}{10} \\
\ganttmilestone[inline=false]{voraussichtliche Abgabe der Bachelor-Arbeit}{10} \\
%\ganttlink{elem4}{elem5}
\end{ganttchart}
}
%\end{landscape}

\end{document}
